\documentclass{beamer}
\usepackage{graphicx} % Required for inserting images

\title{La mia prima presentazione in LaTeX}
\author{Filippo Moratelli}
\date{27 Maggio 2025}

% https://mpetroff.net/files/beamer-theme-matrix/
\usetheme{Singapore} % per cambiare struttura (prefe Duccio è Frankfurt)
\usecolortheme{spruce} % per cambiare colore (prefe Duccio è dove)

\begin{document}

\maketitle

\AtBeginSection[]
{
\begin{frame}{Outline}
\tableofcontents[currentsection]
\end{frame}
} % aggiunge slide automatica che fa vedere in che sezione siamo

\section{Introduction}

\begin{frame}{My first slide}
    This is my first slide in LaTeX.
\end{frame}

\begin{frame}{My second slide}
    This is my second slide in \textbf{LaTeX} and I can make use of items.
    \begin{itemize}
        \item Telerilevamento
        \item \pause Geologia
        \item \pause Ecologia
        \item \pause Geografia % la funz \pause determina ordine di apparizione testo delle slide
    \end{itemize}
\end{frame}

\section{Methods}
\begin{frame}{The formula}
    In this study we made use of the following equations:
    \begin{equation}
        w_{t12}=\int\limits_1^2vdp
    \end{equation}
    
    \bigskip
    This equation has been gathered from:
    \href{https://www.recip.org/tutorial-introduction-to-thermodynamics/}{Introduction to Thermodynamics}.
\end{frame}


\section{Results}
\begin{frame}{Foto singola}
    \includegraphics[width=\linewidth]{DSC_3108.JPG}
\end{frame}

\begin{frame}{Foto doppia}
    \centering
    \includegraphics[width=.5\linewidth]{DSC_3108.JPG}
    \includegraphics[width=.3\linewidth]{DSC_3108.JPG}
\end{frame}

\begin{frame}{4 foto}
    \centering
    \includegraphics[width=.4\linewidth]{DSC_3108.JPG}
    \pause \includegraphics[width=.4\linewidth]{DSC_3108.JPG} \\
    \pause \includegraphics[width=.4\linewidth]{DSC_3108.JPG}
    \pause \includegraphics[width=.4\linewidth]{DSC_3108.JPG}
\end{frame}

\end{document}

% in fondo posso aggiungere commenti
